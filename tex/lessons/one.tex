\section{\q{ }}
\begin{elvish}
\begin{tabularx}{\textwidth}{r@{ }X}
       &      ⸬   \\
჻ &     ⸬ \\
჻   &   \\
჻ &   ⸬ \\
჻   &  ⸱ ⸱    \\
჻  &   \\
჻ &  ⸬  ⸬    \\
჻  & ⸬ \\
჻ &  ⸬ \\
჻  &    ⸬ \\
჻ & ⸬  \\
჻  & 
\end{tabularx}
\end{elvish}

\section{Grammar}
\subsection{Plural formation of nouns}
In \quenya, most grammatical information is contained in the endings of words.
Therefore, along with many other ideas, the plural is expressed by appending an
ending to a noun. In order to determine which one this should be, we can group
the nouns into several classes according to how their basic form ends (the
uninflected nominative). One finds three different groups of nouns:

the first group consists of nouns ending with \q{}, \q{}, \q{} and
\q{} as well as \q{} (in nominative); the second group consists of all
nouns ending in \q{}; the last group is formed by the remaining nouns ending
in a consonant.

Nouns of the first group form plural by appending \q{} to their basic form:
\begin{itemize}
  \item \q{} (\t{tree}) $\rightarrow$ \q{} (\t{trees});
  \item \q{} (\t{friend}) $\rightarrow$ \q{} (\t{friends});
  \item \q{} (\t{queen}) $\rightarrow$ \q{} (\t{queens});
  \item \q{} (\t{path}) $\rightarrow$ \q{} (\t{paths}).
\end{itemize}
The second group forms plural by replacing the final \q{} of the basic form
with \q{}:
\begin{itemize}
  \item \q{} (\t{language}) $\rightarrow$ \q{} (\t{languages});
  \item \q{} (\t{leaf}) $\rightarrow$ \q{} (\t{leaves}).
\end{itemize}
Finally, the third group forms plural by appending \q{} to the final
consonant:
\begin{itemize}
  \item \q{} (\t{king}) $\rightarrow$ \q{} (\t{kings}).
\end{itemize}
Unfortunately, some word involve a complication: these words show a shortened
form in nominative singular which is not identical to the basic form (the stem)
to which endings are appendend. An example of such a word is \q{}
(\t{mountain}) with the stem \q{}.
This means that the plural of \q{} is not \q{} but \q{}.
When we list vocabulary, we list both forms for all words for which stem and
nominative singular are different. The same thing may occur for names, for
example \q{} has the stem \q{}.

\subsection{The definite and indefinite article}
The definite article in \quenya is \q{} in both singular and plural (and the
two other numbers to be discussed in the next lesson as well).
It is placed before the noun:
\begin{itemize}
  \item \q{} (king) $\rightarrow$ \q{ } (the king);
  \item \q{} (trees) $\rightarrow$ \q{ } (the trees);
  \item \q{} (path) $\rightarrow$ \q{ } (the path).
\end{itemize}
For the indefinite article, there is no special word in \quenya~---~it can be
added in the translation when needed:
\begin{itemize}
  \item \q{} (\t{king} or \t{a king}).
\end{itemize}

\subsection{Classes of verbs}
In \quenya, there are basically two main classes of verbs (and several
subclasses). These are usually called basic or primary verbs and derived verbs.
The basic verbs correspond directly to a root in primitive Elvish, whereas the
more numerous derived verbs come from such a root in addition with a
derivational suffix.

Typically, derived verbs can be recognized by the derivational ending, which in
\quenya takes the very typical form \q{}, \q{}, \q{}, \q{} or
sometimes \q{}.
An example might be \q{} (\t{to go}) with the ending \q{} or
\q{} (\t{to stop}) with the ending \q{}. On the other hand, the vast
majority of basic verbs end in a consonant (as most primitive roots do).
Thus, one can easily recognize that \q{} (\t{to wish}, stem) or \q{}
(\t{to speak}, stem) are basic verbs. However, ending in \q{} is no absolute
guarantee that one is dealing with a derived verb~---~but we will deal with the
exceptions later in this course.
The two main classes of verbs show differences when they are conjugated for
different tenses and persons, therefore it is important to recognize and
distinguish them.

Often, something happens to the \emph{stem vowel} of a verb. The term ``stem
vowel'' refers to the vowel which forms part of the primitive root of the verb. For
basic verbs, the stem vowel is often easy to find, because it is the only vowel
of the verb. For derived verbs, the final -a can never be the stem
vowel~---~once this ending is removed, finding the stem vowel in derived verbs 
is easy as well:
E.g., \q{} (\t{to fall}), the \emph{first} \q{} is the stem vowel,
because the last \q{} has to be ruled out. Sometimes, it happens that a verb is prefixed
and after the ending of an derived verb is removed, there are still two different vowels left,
e.g. \q{} (\t{to refill}). In this case, the stem vowel is the last one
after the ending has been removed, i.e. the \q{} of \q{}.

\subsection{The present tense}
The present tense is used in \quenya to express ongoing actions. It is often
used similarly to the present continuous tense (’I am going to
school’), although it is permissible to translate it as present tense as well.
The \quenya present tense is not used to express habits, occupations, timeless
truths and such like.


Basic verbs in \quenya form the present tense by lengthening of the stem vowel
and appending the ending \q{}:
\begin{itemize}
  \item \q{} (\t{to watch}) $\rightarrow$ \q{} (\t{is watching});
  \item \q{} (\t{to come}) $\rightarrow$ \q{} (\t{is coming}).
\end{itemize}
Further endings can be appended to these forms to express person (see below).
For derived verbs, the present tense is formed by replacing the final \q{} by
\q{}. If only one consonant is between this new ending \q{} and the
stem vowel, the stem vowel is lengthened, however if more than one consonant is between
ending and stem vowel, the stem vowel remains short because lengthening is
impossible in such cases. Hence, we find lengthening in:
\begin{itemize}
  \item \q{} (\t{to urge}) $\rightarrow$ \q{} (\t{is urging});
\end{itemize}
but not in:
\begin{itemize}
  \item \q{} (\t{to fall}) $\rightarrow$ \q{} (\t{is falling});
  \item \q{} (\t{to go}) $\rightarrow$ \q{} (\t{is going}).
\end{itemize}
In third age \quenya (especially for derived verbs),
often the Aorist (see lesson 5) is used instead of the present tense since it is
simpler to form. The distinction between these tenses is increasingly blurry in
late \quenya.

\subsection{Pronominal verb endings}
If the subject of a \quenya sentence is given by a separate word, the \quenya
verb in present tense remains in the forms we have discussed above if the word
is in singular. If the subject is in plural, the verb receives an additional
plural ending \q{}:
\begin{itemize}
  \item \q{ } (\t{a leaf is falling});
  \item \q{ } (\t{leaves are falling});
  \item \q{  } (\t{the king is coming});
  \item \q{ } (\t{men are going}).
\end{itemize}
However, in \quenya something special happens when the subject consists of two
persons or things~---~in this case, the \quenya verb assumes the \emph{dual}
verbal endings. In this case, the verb marked by an ending \q{}:
\begin{itemize}
  \item \q{   } (\t{leaf and tree are falling});
  \item \q{     } (\t{the king and the queen are
  coming}).
\end{itemize}
These three usages of the verb in singular, plural and dual with no ending,
\q{} and \q{} are called \emph{personless} verb forms. This means that the
one doing the action described by the verb is not expressed as part of the verb but
stands separately in the sentence.

In contrast, when a pronoun is the subject, this is often expressed in \quenya
by an additional verb ending and not as a separate word. In fact, there are two
different sets of pronominal verb endings in \quenya, long ones and short ones.
We start by listing the short endings in
Table~\ref{tab:pronominal-verb-endings-short}.
See for example:
\begin{itemize}
  \item \q{} (\t{I am coming});
  \item \q{} (\t{You are watching});
  \item \q{} (\t{He is falling});
  \item \q{} (\t{They are going}).
\end{itemize}

\begin{table}
\centering
\caption{Pronominal verb endings (short).}
\label{tab:pronominal-verb-endings-short}
\begin{tabular}{lll}
\toprule
Subject & Singular & Plural \\
\midrule
First person & \q{} (\t{I}) & -- \\
Second person (familiar) & \q{} (\t{thou}) & \q{} (\t{ye}) \\
Second person (formal) & \q{} (\t{you}) & \q{} (\t{you}) \\
Third person & \q{} (\t{he}, \t{she}, \t{it}) & \q{} (\t{they}) \\
\bottomrule
\end{tabular}
\end{table}

Alternatively (and without any difference in meaning) the long endings can be
used. Here, dual forms exist as well (see
Table~\ref{tab:pronominal-verb-endings-long}).
While the two different long endings for second and third persons plural forms
have the same meaning, the two different endings for the first person plural
permit to express a peculiar distinction: the ending \q{} is an
\emph{inclusive} `we', addressing both the group of the speaker and the
listener, it therefore has the meaning of `all of us'; in contrast, the ending
\q{} is an \emph{exclusive} `we', i.e. it makes a distinction between the
group of the speaker and the group of the listener and means `we, but not you'.
If however the speaker refers only to himself and the person addressed, i.e.
`you and me', a dual ending has to be used, either \q{} or more rarely
\q{}. Finally, the dual endings for second and third persons are used used
when two persons or things are referred to, e.g.:
\begin{itemize}
  \item \q{} (\t{the two of you are watching});
  \item \q{} (\t{the pair of them is coming}).
\end{itemize}

\begin{table}
\centering
\caption{Pronominal verb endings (long).}
\label{tab:pronominal-verb-endings-long}
\begin{tabular}{llll}
\toprule
Subject & Singular & Plural & Dual \\
\midrule
First person & \q{} (\t{I}) & \q{}, \q{} (\t{we}) & \q{},
\q{} (\t{we}) \\
Second person (familiar) & \q{} (\t{thou}) & \q{}, \q{} (\t{ye}) &
\q{} (\t{ye}) \\
\\
Second person (formal) & \q{} (\t{you}) & \q{}, \q{} (\t{you}) &
\q{} (\t{you}) \\ \\
Third person (animate) & \q{} (\t{he}, \t{she}) & \q{}, \q{}
(\t{they}) & \q{} (\t{they}) \\
Thir person (inanimate) & \q{} (\t{it}) & \q{}, \q{} (\t{they}) &
\q{} (\t{they}) \\
\bottomrule
\end{tabular}
\end{table}

Examples with long pronominal endings:
\begin{itemize}
  \item \q{} (\t{I'm falling});
  \item \q{} (\t{they are coming});
  \item \q{} (\t{we are watching}).
\end{itemize}
If both subject and object of a sentence are pronouns, it is possible to express
the subject as a long pronominal ending and append the object as a short ending:
\begin{itemize}
  \item \q{} (\t{you see me});
  \item \q{} (\t{I want it});
  \item \q{} (\t{we find them}).
\end{itemize}
However, this cannot be used in expressions like `I am seeing myself', in which
the object refers back to the subject; this requires the reflexive forms.

\subsection{The infinitive}
For basic verbs, the infinitive is forme with the help of the ending \q{}. In
the case of derived verbs, it is identical with the verb stem:
\begin{itemize}
  \item \q{} (\t{to see}) $\rightarrow$ \q{} (\t{to see});
  \item \q{} (\t{to fall}) $\rightarrow$ \q{} (\t{to fall}).
\end{itemize}
The infinitives can be used as the object of a sentence:
\begin{itemize}
  \item \q{ } (\t{I wish to see});
  \item \q{ } (\t{I'm able to speak}).
\end{itemize}

\subsection{The imperative}
The imperative is used to give orders. It is formed by using \q{} or \q{} in
front of the infinitive:
\begin{itemize}
  \item \q{  } (\t{watch the forest!});
  \item \q{ } (\t{go!}).
\end{itemize}
Used like this, the imperative can be used to address a single person or several
persons.

\subsection{The verb \q{}}
The basic for of the verb `to be' in \quenya is \q{} or \q{},
in plural \q{}. Normal pronominal endings can be used with this verb,
presumably all endings are appended to the short \q{} and not to the long
\q{} (see Table~\ref{tab:na-verb-conjugation}).

\begin{table}
\centering
\caption{Conjugation of \q{} (common endings).}
\begin{tabular}{lll}
\toprule
Subject & Singular & Plural \\
\midrule
First person & \q{} (\t{I am}) & \q{} (\t{we are}) \\
Second person & \q{} (\t{you are}) & \q{} (\t{you are}) \\
Third person & \q{} (\t{he/she/it is}) & \q{} (\t{they are}) \\
\bottomrule
\end{tabular}
\end{table}

The verb \q{} is often left out when it is understood, and, if present,
usually moves towards the end of the sentence:
\begin{itemize}
  \item \q{  } (\t{\q{} is a \q{}}).
\end{itemize}

\subsection{Questions}
An ordinary sentence can be transformed into a question by adding the word
\q{} in front of the sentence:
\begin{itemize}
  \item \q{  } (\t{do you see a man?});
  \item \q{     } (\t{Is there a \q{} beneath the tree?}).
\end{itemize}

\section{Vocabulary}
\begin{center}
\begin{tabular}{ll|ll}
\toprule
\q{}   & \t{tree}       & \q{}    & \t{where} \\
\q{}   & \t{and}        & \q{}    & \t{friend} \\
\q{}  & \t{king}       & \q{}   & \t{to wish}, \t{to desire} \\
\q{}   & \t{again}      & \q{} & \t{farewell} \\
\q{}  & \t{to see}     & \q{}   & \t{friendship} \\
\q{} & \t{to refill}  & \q{} (\q{}) & \t{man} \\
\q{} & \t{to stop}    & \q{}     & \t{under}, \t{beneath} \\
\q{}  & \t{to follow}  & \q{}   & \t{meeting} \\
\q{}  & \t{to find}    & \q{}   & \t{to urge}, \t{to impel} \\
\q{}   & \t{between}, \t{among}     & \q{}   & \t{can}, \t{to be able
to} \\
\q{}   & \t{tongue}, \t{language}   & \q{}   & \t{to speak} \\
\q{} (\q{})  & \t{to fall} & \q{}   & \t{here} \\
\q{}   & \t{leaf}       & \q{}   & \t{queen} \\
\q{}  (\q{}) & \t{to go}, \t{to travel} & \q{} & \t{forest} \\
\q{} & \t{shadow}     & \q{}    & \t{path}, \t{way} \\
\q{}   & \t{who}, \t{what} & \q{}   & \t{to watch} \\
\q{}  & \t{good}, \t{well}& \q{}   & \t{to come} \\
\bottomrule
\end{tabular}
\end{center}

\section{\q{}}