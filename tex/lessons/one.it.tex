\section{\q{ }}
\begin{elvish}
\begin{tabularx}{\textwidth}{r@{ }X}
       &      ⸬   \\
჻ &     ⸬ \\
჻   &   \\
჻ &   ⸬ \\
჻   &  ⸱ ⸱    \\
჻  &   \\
჻ &  ⸬  ⸬    \\
჻  & ⸬ \\
჻ &  ⸬ \\
჻  &    ⸬ \\
჻ & ⸬  \\
჻  & 
\end{tabularx}
\end{elvish}

\section{Grammatica}
\subsection{Formazione del plurale di nomi}
In \quenya, la maggior parte dell'informazione grammaticale è contenuta
nelle desinenze delle parole. Dunque, insieme a molte altre forme, il plurale si
esprime appendenendo una desinenza ad un nome. Per determinare quale desinenza
debba essere utilizzata, possiamo raggruppare i nomi in diverse classi a seconda
di come terminano le loro forme base (il nominativo non inflesso). Si possono
trovare tre differenti gruppi di nomi: il primo gruppo consiste dei nomi che
terminano in \q{}, \q{}, \q{} e \q{} così come \q{} (al nominativo);
il secondo gruppo consiste di tutti i nomi che terminano in \q{}; il terzo
gruppo è infine formato dai restanti nomi che terminano con una consonante.

Nomi del primo gruppo formano il plurale appendendo \q{} alla loro forma base:
\begin{itemize}
  \item \q{} (albero) $\rightarrow$ \q{} (alberi);
  \item \q{} (amico) $\rightarrow$ \q{} (amici);
  \item \q{} (regina) $\rightarrow$ \q{} (regine);
  \item \q{} (cammino) $\rightarrow$ \q{} (cammini).
\end{itemize}
Il secondo gruppo forma il plurale sostituendo la \q{} finale della forma base
con una \q{}:
\begin{itemize}
  \item \q{} (lingua) $\rightarrow$ \q{} (lingue);
  \item \q{} (foglia) $\rightarrow$ \q{} (foglie).
\end{itemize}
Infine, il terzo gruppo forma il plurale appendendo una \q{} alla consonante
finale:
\begin{itemize}
  \item \q{} (re) $\rightarrow$ \q{} (re).
\end{itemize}
Sfortunatamente, alcune parole presentano una complicazione: queste parole hanno
una forma abbreviata al nominativo singolare che non è uguale alla forma base
(la radice) a cui le desinenze sono appese. Un esempio è \q{} (montagna) la
cui radice è \q{}. Ciò significa che la forma plurale di \q{} non è
\q{} ma \q{}.
Nella lista dei vocaboli sono indicate entrambe le forme per tutti i termini la
cui radice differisce dal nominativo singolare.

\subsection{Articolo determinativo e indeterminativo}
L'articolo determinativo in \quenya è \q{} sia al singolare che al plurale
(così come nelle altre due forme che saranno discusse nella prossima lezione). È posto
prima del nome:
\begin{itemize}
  \item \q{} (re) $\rightarrow$ \q{ } (il re);
  \item \q{} (alberi) $\rightarrow$ \q{ } (gli alberi);
  \item \q{} (cammino) $\rightarrow$ \q{ } (il cammino).
\end{itemize}
L'articolo indeterminativo non ha un corrispettivo in \quenya~---~può essere
semplicemente aggiunto nella traduzione se necessario:
\begin{itemize}
  \item \q{} (re \emph{oppure} un re).
\end{itemize}

\subsection{Classi di verbi}
In \quenya, ci sono fondamentalmente due classi principali di verbi,
chiamate ``verbi base o primari'' e ``verbi derivati''. I verbi base sono
costituiti soltanto dalla loro radice primitiva, mentre i verbi derivati oltre
che dalla radice sono costituiti anche da un suffisso di derivazione.

Tipicamente, i verbi derivati possono essere riconosciuti dalla
desinenza di derivazione, che in \quenya ha la forma molto tipica
\q{}, \q{}, \q{}, \q{} o qualche volta \q{}.
Un esempio è \q{} (\t{andare}) con la desinenza \q{} oppure \q{}
(\t{fermare}) con la desinenza \q{}.
D'altra parte la vasta maggioranza dei verbi base termina in consonante (come
la maggior parte delle radici) e di conseguenza è facile riconoscere che \q{}
(\t{desiderare}) o \q{} (\t{parlare}) sono verbi base. Non è comunque
garantito che un verbo che termina in \q{} sia un verbo derivato~---~ma
tratteremo le eccezioni più avanti in questo corso.
Le due principali classi di verbi mostrano differenze quando sono coniugati per
diversi tempi e persone, per questo è importante imparare a distinguerli.

Spesso accade qualcosa alla \emph{vocale della radice} di un verbo. Per
``vocale della radice'' si intende la vocale che fa parte della radice primitiva
del verbo. Per i verbi base, la vocale della radice è facile da individuare,
perché è l'unica vocale del verbo. Per i verbi derivati, la \q{} finale non
può mai essere la vocale della radice~---~una volta rimossa, trovare la vocale
della radice dei verbi derivati diventa ugualmente facile: ad esempio in
\q{} (\t{cadere}), la \emph{prima} \q{} è la vocale della radice, perché l'ultima
\q{} deve essere esclusa. A volte può capitare che un verbo abbia un prefisso
e dopo aver rimosso la \q{} finale di un verbo derivato ci siano ancora due
diverse vocali rimaste, ad esempio in \q{} (\t{riempire}). In questo caso, la
vocale della radice è l'ultima vocale una volta che la \q{} finale è stata
rimossa.
