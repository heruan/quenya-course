\section{\q{ }}
\begin{elvish}
\begin{tabularx}{\textwidth}{r@{ }X}
       &      ⸬   \\
჻ &     ⸬ \\
჻    &   \\
჻ &   ⸬ \\
჻    &  ⸱ ⸱    \\
჻  &   \\
჻ &  ⸬  ⸬    \\
჻  & ⸬ \\
჻ &  ⸬ \\
჻  &    ⸬ \\
჻ & ⸬  \\
჻  & 
\end{tabularx}
\end{elvish}

\section{Grammatica}
\subsection{Formazione del plurale dei sostantivi}
In \quenya, la maggior parte dell'informazione grammaticale è contenuta nelle
desinenze delle parole. Pertanto, così come molti altri concetti, il plurale si
esprime aggiungendo una desinenza a un sostantivo. Al fine di determinare quale
dovrebbe essere, possiamo raggruppare i sostantivi in diverse classi a seconda
di come termina la loro forma base (il nominatino non inflesso).
Si trovano tre diversi gruppi di sostantivi:
il primo gruppo è composto da sostantivi che terminano con \q{}, \q{},
\q{} e \q{} così come \q{} (al nominativo); il secondo gruppo è
composto da tutti i sostantivi che terminano in \q{}; l'ultimo gruppo è
formato dai restanti sostantivi che terminano con una consonante.

I sostantivi del primo formano il plurale aggiungendo \q{} alla loro forma
base:
\begin{itemize}
  \item \q{} (\t{albero}) $\rightarrow$ \q{} (\t{alberi});
  \item \q{} (\t{amico}) $\rightarrow$ \q{} (\t{amici});
  \item \q{} (\t{regina}) $\rightarrow$ \q{} (\t{regine});
  \item \q{} (\t{via}) $\rightarrow$ \q{} (\t{vie}).
\end{itemize}
Il secondo gruppo forma il plurale sostituendo la \q{} finale della loro
forma base con \q{}:
\begin{itemize}
  \item \q{} (\t{lingua}) $\rightarrow$ \q{} (\t{lingue});
  \item \q{} (\t{foglia}) $\rightarrow$ \q{} (\t{foglie}).
\end{itemize}
Infine, il terzo gruppo forma il plurale aggiungendo \q{} alla consonante
finale:
\begin{itemize}
  \item \q{} (un \t{re}) $\rightarrow$ \q{} (più \t{re}).
\end{itemize}
Purtroppo, qualche parola comporta una complicazione: queste parole mostrano una
forma abbreviata al nominativo singolare che non è identica alla forma base
(radice). Un esempio di tale parola è \q{} (\t{montagna}) con radice
\q{}.
Ciò significa che il plurale di \q{} non è \q{} ma \q{}.
Quando elenchiamo i vocaboli, riportiamo entrambe entrambe le forme per tutte le
parole la cui radice differisce dal nominativo singolare. La stessa cosa può
verificarsi per i nomi propri, ad esempio \q{} ha radice \q{}.

\subsection{L'articolo determinativo}
L'articolo determinativo in \quenya è \q{} sia al singolare che al plurale
(e negli altri due numeri che saranno discussi nella prossima lezione).
Si posiziona prima del sostantivo:
\begin{itemize}
  \item \q{} (re) $\rightarrow$ \q{ } (il re);
  \item \q{} (alberi) $\rightarrow$ \q{ } (gli alberi);
  \item \q{} (via) $\rightarrow$ \q{ } (la via).
\end{itemize}
Per l'articolo indeterminativo non c'è una parola speciale in \quenya~---~può
essere utilizzato nella traduzione quando necessario:
\begin{itemize}
  \item \q{} (\t{re} oppure \t{un re}).
\end{itemize}

\subsection{Classi di verbi}
In \quenya ci sono fondamentalmente due classi principali di verbi (e diverse
sottoclassi). Queste sono generalmente chiamate verbi base o primari e verbi
derivati. I verbi base corrispondono direttamente a una radice in Elfico
primitivo, mentre i più numerosi verbi derivati provengono da tale radice con
l'aggiunta di un suffisso.

Tipicamente i verbi derivati possono essere riconosciuti dalla desinenza, che
in \quenya assume le forme molto tipiche \q{}, \q{}, \q{}, \q{} o
a volte \q{}.
Un esempio potrebbe essere \q{} (\t{andare}) con la desinenza \q{}
oppure \q{} (\t{fermare}) con la desinenza \q{}. D'altra parte, la
stragrande maggioranza dei verbi base termina con una consonante (come la
maggior parte delle radici primitive).
Così, si può facilmente riconoscere che \q{} (\t{desiderare}) o
\q{} (\t{parlare}) sono verbi base. Tuttavia, la desinenza \q{} non è
garanzia assoluta che si tratti di un verbo derivato~---~ma ci si occuperà
delle eccezioni più avanti in questo corso.
Le due classi principali di verbi mostrano differenze quando vengono coniugati
per diversi tempi e persone, è pertanto importante saperli riconoscere e
distinguere tra loro.

Spesso, succede qualcosa alla \emph{radice vocalica} di un verbo. Il termine
``radice vocalica'' si riferisce alla vocale che fa parte della radice
primitiva del verbo.
Per i verbi base, la radice vocalica è spesso facile da trovare, perché è la
sola vocale del verbo.
Per i verbi derivati, la\q{} finale non può mai essere la radice
vocalica~---~una volta che questa desinenza viene rimossa, trovare la radice
vocalica nei verbi derivati diventa semplice:
ad esempio, \q{} (\t{cadere}), la \emph{prima} \q{} è la radice vocalica,
perché l'ultima \q{} deve essere esclusa.
A volte, capita che un vervo abbia un prefisso e dopo che la desinenza è
rimossa, ci sono ancora due diverse vocali rimaste; ad esempio \q{}
(\t{riempire}).
In questo caso, la radice vocalica è l'ultima dopo che la desinenza è stata
rimossa, ovvero la \q{} di \q{}.

\subsection{Il tempo presente}
Il tempo presente è utilizzato in \quenya per esprimere azioni in corso.

I verbi base in \quenya formano il tempo presente con l'allungamento della
radice vocalica e aggiungendo la desinenza \q{}:
\begin{itemize}
  \item \q{} (\t{guardare}) $\rightarrow$ \q{} (\t{sta guardando});
  \item \q{} (\t{venire}) $\rightarrow$ \q{} (\t{sta venendo}).
\end{itemize}
Ulteriori desinenze possono essere aggiunte a questa forma per esprimere la
persona (vedi più avanti).
Per i verbi derivati, il tempo presente è formato sostituendo la \q{} finale
con \q{}. Se c'è solo una consonante tra questa nuova desinenza \q{}
e la radice vocalica, la radice vocalica è allungata, se invece c'è più di una
consonante tra la desinenza e la radice vocalica, la radice vocalica rimane
breve perché l'allungamento in questi casi è impossibile.
Quindi, troviamo allungamento in:
\begin{itemize}
  \item \q{} (\t{esortare}) $\rightarrow$ \q{} (\t{sta esortando});
\end{itemize}
ma non in:
\begin{itemize}
  \item \q{} (\t{cadere}) $\rightarrow$ \q{} (\t{sta cadendo});
  \item \q{} (\t{andare}) $\rightarrow$ \q{} (\t{sta andando}).
\end{itemize}
Nel \quenya della terza era (specialmente nei verbi derivati),
è usato spesso l'Aoristo (vedi Capitolo~5) al posto del presente essendo più
semplice da formare.
Le distinzione tra questi tempi è sempre più indistinta nel tardo \quenya.

\subsection{Pronominal verb endings}
If the subject of a \quenya sentence is given by a separate word, the \quenya
verb in present tense remains in the forms we have discussed above if the word
is in singular. If the subject is in plural, the verb receives an additional
plural ending \q{}:
\begin{itemize}
  \item \q{ } (\t{a leaf is falling});
  \item \q{ } (\t{leaves are falling});
  \item \q{  } (\t{the king is coming});
  \item \q{ } (\t{men are going}).
\end{itemize}
However, in \quenya something special happens when the subject consists of two
persons or things~---~in this case, the \quenya verb assumes the \emph{dual}
verbal endings. In this case, the verb marked by an ending \q{}:
\begin{itemize}
  \item \q{   } (\t{leaf and tree are falling});
  \item \q{     } (\t{the king and the queen are
  coming}).
\end{itemize}
These three usages of the verb in singular, plural and dual with no ending,
\q{} and \q{} are called \emph{personless} verb forms. This means that the
one doing the action described by the verb is not expressed as part of the verb but
stands separately in the sentence.

In contrast, when a pronoun is the subject, this is often expressed in \quenya
by an additional verb ending and not as a separate word. In fact, there are two
different sets of pronominal verb endings in \quenya, long ones and short ones.
We start by listing the short endings in
Table~\ref{tab:pronominal-verb-endings-short}.
See for example:
\begin{itemize}
  \item \q{} (\t{I am coming});
  \item \q{} (\t{You are watching});
  \item \q{} (\t{He is falling});
  \item \q{} (\t{They are going}).
\end{itemize}

\begin{table}
\centering
\caption{Pronominal verb endings (short).}
\label{tab:pronominal-verb-endings-short}
\begin{tabular}{lll}
\toprule
Subject & Singular & Plural \\
\midrule
First person & \q{} (\t{I}) & -- \\
Second person (familiar) & \q{} (\t{thou}) & \q{} (\t{ye}) \\
Second person (formal) & \q{} (\t{you}) & \q{} (\t{you}) \\
Third person & \q{} (\t{he}, \t{she}, \t{it}) & \q{} (\t{they}) \\
\bottomrule
\end{tabular}
\end{table}

Alternatively (and without any difference in meaning) the long endings can be
used. Here, dual forms exist as well (see
Table~\ref{tab:pronominal-verb-endings-long}).
While the two different long endings for second and third persons plural forms
have the same meaning, the two different endings for the first person plural
permit to express a peculiar distinction: the ending \q{} is an
\emph{inclusive} `we', addressing both the group of the speaker and the
listener, it therefore has the meaning of `all of us'; in contrast, the ending
\q{} is an \emph{exclusive} `we', i.e. it makes a distinction between the
group of the speaker and the group of the listener and means `we, but not you'.
If however the speaker refers only to himself and the person addressed, i.e.
`you and me', a dual ending has to be used, either \q{} or more rarely
\q{}. Finally, the dual endings for second and third persons are used used
when two persons or things are referred to, e.g.:
\begin{itemize}
  \item \q{} (\t{the two of you are watching});
  \item \q{} (\t{the pair of them is coming}).
\end{itemize}
\begin{table}
\centering
\caption{Pronominal verb endings (long).}
\label{tab:pronominal-verb-endings-long}
\begin{tabular}{llll}
\toprule
Subject & Singular & Plural & Dual \\
\midrule
First person & \q{} (\t{I}) & \q{}, \q{} (\t{we}) & \q{},
\q{} (\t{we}) \\
Second person (familiar) & \q{} (\t{thou}) & \q{}, \q{} (\t{ye}) &
\q{} (\t{ye}) \\
\\
Second person (formal) & \q{} (\t{you}) & \q{}, \q{} (\t{you}) &
\q{} (\t{you}) \\ \\
Third person (animate) & \q{} (\t{he}, \t{she}) & \q{}, \q{}
(\t{they}) & \q{} (\t{they}) \\
Thir person (inanimate) & \q{} (\t{it}) & \q{}, \q{} (\t{they}) &
\q{} (\t{they}) \\
\bottomrule
\end{tabular}
\end{table}
Examples with long pronominal endings:
\begin{itemize}
  \item \q{} (\t{I'm falling});
  \item \q{} (\t{they are coming});
  \item \q{} (\t{we are watching}).
\end{itemize}
If both subject and object of a sentence are pronouns, it is possible to express
the subject as a long pronominal ending and append the object as a short ending:
\begin{itemize}
  \item \q{} (\t{you see me});
  \item \q{} (\t{I want it});
  \item \q{} (\t{we find them}).
\end{itemize}
However, this cannot be used in expressions like `I am seeing myself', in which
the object refers back to the subject; this requires the reflexive forms.

\subsection{The infinitive}
For basic verbs, the infinitive is forme with the help of the ending \q{}. In
the case of derived verbs, it is identical with the verb stem:
\begin{itemize}
  \item \q{} (\t{to see}) $\rightarrow$ \q{} (\t{to see});
  \item \q{} (\t{to fall}) $\rightarrow$ \q{} (\t{to fall}).
\end{itemize}
The infinitives can be used as the object of a sentence:
\begin{itemize}
  \item \q{ } (\t{I wish to see});
  \item \q{ } (\t{I'm able to speak}).
\end{itemize}

\subsection{The imperative}
The imperative is used to give orders. It is formed by using \q{} or \q{} in
front of the infinitive:
\begin{itemize}
  \item \q{  } (\t{watch the forest!});
  \item \q{ } (\t{go!}).
\end{itemize}
Used like this, the imperative can be used to address a single person or several
persons.

\subsection{The verb \q{}}
The basic for of the verb `to be' in \quenya is \q{} or \q{},
in plural \q{}. Normal pronominal endings can be used with this verb,
presumably all endings are appended to the short \q{} and not to the long
\q{} (see Table~\ref{tab:na-verb-conjugation}).

\begin{table}
\centering
\caption{Conjugation of \q{} (common endings).}
\begin{tabular}{lll}
\toprule
Subject & Singular & Plural \\
\midrule
First person & \q{} (\t{I am}) & \q{} (\t{we are}) \\
Second person & \q{} (\t{you are}) & \q{} (\t{you are}) \\
Third person & \q{} (\t{he/she/it is}) & \q{} (\t{they are}) \\
\bottomrule
\end{tabular}
\end{table}

The verb \q{} is often left out when it is understood, and, if present,
usually moves towards the end of the sentence:
\begin{itemize}
  \item \q{  } (\t{\q{} is a \q{}}).
\end{itemize}

\subsection{Questions}
An ordinary sentence can be transformed into a question by adding the word
\q{} in front of the sentence:
\begin{itemize}
  \item \q{  } (\t{do you see a man?});
  \item \q{     } (\t{Is there a \q{} beneath the tree?}).
\end{itemize}

\section{Vocabulary}
\begin{center}
\begin{tabular}{ll|ll}
\toprule
\q{}   & \t{tree}       & \q{}    & \t{where} \\
\q{}   & \t{and}        & \q{}    & \t{friend} \\
\q{}  & \t{king}       & \q{}   & \t{to wish}, \t{to desire} \\
\q{}   & \t{again}      & \q{} & \t{farewell} \\
\q{}  & \t{to see}     & \q{}   & \t{friendship} \\
\q{} & \t{to refill}  & \q{} (\q{}) & \t{man} \\
\q{} & \t{to stop}    & \q{}     & \t{under}, \t{beneath} \\
\q{}  & \t{to follow}  & \q{}   & \t{meeting} \\
\q{}  & \t{to find}    & \q{}   & \t{to urge}, \t{to impel} \\
\q{}   & \t{between}, \t{among}     & \q{}   & \t{can}, \t{to be able
to} \\
\q{}   & \t{tongue}, \t{language}   & \q{}   & \t{to speak} \\
\q{} (\q{})  & \t{to fall} & \q{}   & \t{here} \\
\q{}   & \t{leaf}       & \q{}   & \t{queen} \\
\q{}  (\q{}) & \t{to go}, \t{to travel} & \q{} & \t{forest} \\
\q{} & \t{shadow}     & \q{}    & \t{path}, \t{way} \\
\q{}   & \t{who}, \t{what} & \q{}   & \t{to watch} \\
\q{}  & \t{good}, \t{well}& \q{}   & \t{to come} \\
\bottomrule
\end{tabular}
\end{center}

\section{\q{}}